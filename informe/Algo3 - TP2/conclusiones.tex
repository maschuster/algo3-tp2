\section{Conclusiones}

En este trabajo se presentaron distintas técnicas para resolver PCMI, que resulta aplicable a problemas de la vida real. Se corrieron los algoritmos desarrollados para instancias de distintos tamaños, analizando y comparando el tiempo de ejecución y la calidad de las soluciones encontradas.

Se entrenaron los parámetros óptimos de Tabú Search (TS) en un subconjunto de instancias de \textit{train}, y se corroboraron en instancias de \textit{test} obteniendo buenos resultados.

Se pudo concluir que los algoritmos golosos son buenos para el costo computacional que implican, pero vale la pena invertir el tiempo extra por la calidad que brinda TS. Se probó que TS puede resolver el problema de forma satisfactoria cuando la generación de la vecindad emplea la estrategia de change. Cabe destacar que en este caso es importante emplear como heurística inicial a W o WP.

Sería interesante repetir la experimentación en un número de instancias mayor para poder determinar con mayor seguridad el ranking de las diferentes implementaciones de TSC que dieron un gap relativo muy similar.